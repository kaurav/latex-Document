\section{SageMath}

\begin{figure}[!ht]
\centering
\includegraphics[width=0.7\textwidth]{images/sage.png}                   
\caption{SageMath Logo}
\hspace{-1.5em}
\end{figure}

SageMath (previously Sage or SAGE, System for Algebra and Geometry Experimentation) is mathematical software with features covering many aspects of mathematics, including algebra, combinatorics, numerical mathematics, number theory, and calculus.


The first version of SageMath was released on 24 February 2005 as free and open source software under the terms of the GNU General Public License, with the initial goals of creating an "open source alternative to Magma, Maple, Mathematica, and MATLAB". The originator and leader of the SageMath project, William Stein, is a mathematician at the University of Washington.


SageMath uses the Python programming language, supporting procedural, functional and object-oriented constructs.


\subsection{Features}
The Sage notebook document interface in a web browser.
Equation solving and typesetting using the SageMath notebook web interface


Features of SageMath include:
\begin{itemize}
\item  A browser-based notebook for review and re-use of previous inputs and outputs, including graphics and text annotations. Compatible with Firefox, Opera, Konqueror, Google Chrome and Safari. Notebooks can be accessed locally or remotely and the connection can be secured with HTTPS.
\item   A text-based command-line interface using IPython
\item   Support for parallel processing using multi-core processors, multiple processors, or distributed computing
\item   Calculus using Maxima and SymPy
\item   Numerical linear algebra using the GSL, SciPy and NumPy
\item   2D and 3D graphs of symbolic functions and numerical data
\item   Matrix manipulation, including sparse arrays
\item  Multivariate statistics libraries, using R and SciPy
\item A toolkit for adding user interfaces to calculations and applications
\item Graph theory visualization and analysis tools
\item Libraries of number theory functions
\item Support for complex numbers, arbitrary precision and symbolic computation
\item Technical word processing including formula editing and embedding SageMath within LaTeX documents
\item The Python standard library, including tools for connecting to SQL, HTTP, HTTPS, NNTP, IMAP, SSH, IRC, FTP and others
\item Interfaces to some third-party applications like Mathematica, Magma, R, and Maple
\item Documentation using Sphinx
\item An automated test-suite
\item Execution of Fortran, C, C++, and Cython code
\item Although not provided by SageMath directly, SageMath can be called from within Mathematica as is done in this Mathematica notebook example
\end{itemize}

\subsection{Installation From Source Code}
Steps for install SageMath -:
\begin{itemize}
\item	sudo apt-get install g++ lrzip
\item	lrunzip  "Downloaded-Zip-file".tar.lrz
\item	sudo apt-get dpkg-dev
\item	cd sageL
\item	sudo apt-get install binutils gcc make m4 perl tar
\item	sudo apt-get install build-essential m4
\item	export MAKE='make -j8'
\item	sudo apt-get install tk8.5-dev
\item	cd sage-directory
\item	./configure
\item	make
\end{itemize}

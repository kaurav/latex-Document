\section{Conclusion}
DoS is a very efficient application which help in generating result for analysis of structures. It
can be used by Civil Engineers and M.Tech. students and even layman. It's less time consuming and user-friendly which let the User work in batch mode 
and free him from all the installation process. It is a web application that can be accessed from a number of devices. The responsive User Interface makes it easy for the users to operate it. Many efforts were made to ease the usage for the users. Hence, it is expected to be work properly in different conditions. But any future bug reports or improvements are always welcomed and will be processed happily.

I learn a lot by working on this project. During this period I got to learn a vast number of
technologies. These are listed below:
Operating system: Ubuntu
Language used: Python, HTML, CSS, shellscript
Framework: Django
Technogloy: \LaTeX{}, Djanog, Doxygen, Sagemath, Git 
So during this project I learn  all the above things. Above all I got to know how software is
developed and how much work and attention to details is required in building even the most basic
of components of any project. Planning, designing, developing code, working in a team, testing,
etc. These are all very precious lessons in themselves.
Aside from all above I got go know about various methods like -:
\begin{enumerate}
\item Threading the programs 
\item Embedding and using different tech in one software.
\item How to work like in group for development of software.
\item How to apply juggaar(innovated) in softwares to get problem solved. 
\end{enumerate}

Beside these technology used in project I also get to know some other tech also like -:
\begin{enumerate}
\item  OpneCV (image processing)
\item opensshserver 
\item reveal.md, impress.js (for making presentations)
\end{enumerate}  


